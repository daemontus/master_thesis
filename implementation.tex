The algorithm presented in this work is available as an open source implementation [CITE]. This implementation forms a key part in the Pithya parameter synthesis tool for ODE based biochemical models.

In this chapter, we discuss the architecture and characteristics of this implementation.

\section{Pithya core overview}

The Pithya tool has two main components: \emph{graphical user interface} and the \emph{core engine}. In this work, we are concerned only about the core engine.

The core engine is implemented in an object-oriented manner using the Kotlin programming language (compiles to standard JVM byte-code). Furthermore, the engine can use the Z3 SMT solver for decisions about the parameter formulae.

The core engine itself is also divided into several modules:

\begin{itemize}
	\item \textbf{Temporal Logic Module} This module is responsible for parsing the \ac{HUCTLp} formulae and performing necessary transformations to ensure the formulae use only the supported set of operators. The input format of the \ac{HUCTLp} formulae is specified as an ATLR4 grammar.
	\item \textbf{Parameter Synthesis Module} The main module containing the algorithm itself with abstract definitions of the necessary data structures such as solver, state map or model.
	\item \textbf{ODE Model Module} Defines a parser for the \texttt{.bio} ODE model files and a set of solvers, successor generators and state maps that work with ODE models.
	\item \textbf{CLI Front-end} Provides a command line interface, combining the functionality of all modules into one executable.
\end{itemize}

In the following sections we will discuss in detail the specified modules. We will omit the temporal logic module, since its only interesting part is the \ac{HUCTLp} grammar. This grammar can be found in Appendix X.

\section{Parameter Synthesis Module}

