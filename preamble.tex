\documentclass[
  digital, %% This option enables the default options for the
           %% digital version of a document. Replace with `printed`
           %% to enable the default options for the printed version
           %% of a document.
  table,   %% Causes the coloring of tables. Replace with `notable`
           %% to restore plain tables.
  lof,     %% Prints the List of Figures. Replace with `nolof` to
           %% hide the List of Figures.
  lot,     %% Prints the List of Tables. Replace with `nolot` to
           %% hide the List of Tables.
  %% More options are listed in the user guide at
  %% <http://mirrors.ctan.org/macros/latex/contrib/fithesis/guide/mu/fi.pdf>.
]{fithesis3}
%% The following section sets up the locales used in the thesis.
\usepackage[resetfonts]{cmap} %% We need to load the T2A font encoding
\usepackage[T1,T2A]{fontenc}  %% to use the Cyrillic fonts with Russian texts.
\usepackage[
  main=english, %% By using `czech` or `slovak` as the main locale
                %% instead of `english`, you can typeset the thesis
                %% in either Czech or Slovak, respectively.
  german, russian, czech, slovak %% The additional keys allow
]{babel}        %% foreign texts to be typeset as follows:
\usepackage{acro}
%%
%%   \begin{otherlanguage}{german}  ... \end{otherlanguage}
%%   \begin{otherlanguage}{russian} ... \end{otherlanguage}
%%   \begin{otherlanguage}{czech}   ... \end{otherlanguage}
%%   \begin{otherlanguage}{slovak}  ... \end{otherlanguage}
%%
%% For non-Latin scripts, it may be necessary to load additional
%% fonts:
\usepackage{paratype}
\def\textrussian#1{{\usefont{T2A}{PTSerif-TLF}{m}{rm}#1}}
%%
%% The following section sets up the metadata of the thesis.
\thesissetup{
    date          = \the\year/\the\month/\the\day,
    university    = mu,
    faculty       = fi,
    type          = mgr,
    author        = Samuel Pastva,
    gender        = m,
    advisor       = {prof. RNDr. Luboš Brim, CSc.},
    title         = {Parallel parameter synthesis for the HUCTL logic},
    TeXtitle      = {Parallel parameter synthesis for the $HUCTL$ logic},
    keywords      = {TODO, ...},
    TeXkeywords   = {TODO, \ldots},
    bib           = main.bib,
}

\usepackage{algorithmicx}

\usepackage{makeidx}      %% The `makeidx` package contains
\makeindex                %% helper commands for index typesetting.
%% These additional packages are used within the document:
\usepackage{paralist} %% Compact list environments
\usepackage{amsmath}  %% Mathematics
\usepackage{amsthm}
\usepackage{amsfonts}
\usepackage{url}      %% Hyperlinks

% Custom macros and definitions

\newcommand{\INIT}{\STATE \textbf{init} {}}
\newcommand{\df}{\scalebox{.9}{$\stackrel{{\tiny \mathrm{df}}}{=}$}}
\newcommand{\TBA}{{\bf TBA}}

\newcommand{\ptort}[1]{\mbox{$\stackrel{#1}{\rightarrow}\hspace{-.65ex}%
		\raisebox{.15em}{\scriptsize *}\hspace{.65ex}$}}
\newcommand{\ptot}[1]{\mbox{$\stackrel{#1}{\rightarrow}\hspace{-.65ex}%
		\raisebox{.15em}{\scriptsize +}\hspace{.65ex}$}}
\newcommand{\pto}[1]{\stackrel{#1}{\rightarrow}}
\newcommand{\ptoprime}[1]{\stackrel{#1}{\rightarrow}\hspace{-1mm}\raisebox{-0.5mm}{$'$}{\,}}
\newcommand{\redmod}[2]{{#1}|_{#2}}
\DeclareMathOperator{\Succ}{Succ}
\DeclareMathOperator{\SCC}{SCC}
\newcommand{\abs}[1]{\lvert #1\rvert}

\newcommand{\ks}[1][]{\ensuremath{ \mathcal K_{#1}=(\mathcal P, S_{#1},I_{#1},\to_{#1}, L_{#1})\ }}
%{{{ Mod_M
\renewcommand{\mod}[1][\mathcal K]
{\ensuremath{Fragment_{#1}}}
\newcommand{\modul}[1][\mathcal K]%
{\ensuremath{\mathcal{F}_{#1}}}
%}}}

%{{{ MC(M,As) (\mc structure as.f.)
\newcommand{\mc}[1][\mathcal K]{\ensuremath{\mathcal{C}_{#1}}}
\newcommand{\mcpsi}[1][\mathcal K]{\ensuremath{\mathcal{C}_{#1}^\psi}}
%}}}

%{{{ True, false
%\newcommand{\ttrue}{\ensuremath{\texttt{tt}}}
%\newcommand{\ffalse}{\ensuremath{\texttt{ff}}}
%}}}
%{{{ assumption function
\newcommand{\as}[1][]{\ensuremath{\mathcal{A}_{#1}}}
\newcommand{\AS}[1][\mathcal K]{\ensuremath{AS_{#1}}}
\newcommand{\ASpsi}{\ensuremath{AS_{\mathcal K}^\psi}}
\newcommand{\asu}{\ensuremath{\mathcal{A}_{\perp}}}
%}}}

\newcommand{\bind}{\downarrow\hspace*{-.5ex}}

\usepackage{algorithm}
\usepackage[noend]{algpseudocode}
% New definitions
\algnewcommand\algorithmicswitch{\textbf{switch}}
\algnewcommand\algorithmiccase{\textbf{case}}
\algnewcommand\algorithmicassert{\texttt{assert}}
\algnewcommand\Assert[1]{\State \algorithmicassert(#1)}%
% New "environments"
\algdef{SE}[SWITCH]{Switch}{EndSwitch}[1]{\algorithmicswitch\ #1\ \algorithmicdo}{\algorithmicend\ \algorithmicswitch}%
\algdef{SE}[CASE]{Case}{EndCase}[1]{\algorithmiccase\
	#1}{\algorithmicend\ \algorithmiccase}%


\algblockdefx{FORALLP}{ENDFAP}[2]%
{\textbf{for all }#1 \textbf{do in parallel} #2}%

%% auxiliary macros for CTL formulae
\newcommand{\TLfont}[1]{\mathbf{#1}}
\newcommand{\actl}[2]{{}_\mathnormal{#2}{#1}}
\newcommand{\actlr}[3]{{}_\mathnormal{#2}{#1}_\mathnormal{#3}}
\newcommand{\TLop}[1]{\operatorname{\TLfont{{#1}}}}
\newcommand{\TLbin}[1]{\mathbin{\TLop{#1}}}
\newcommand{\CTLop}[2]{\operatorname{\TLfont{#1{#2}}}}
\newcommand{\TLopl}[2]{\operatorname{\TLfont{\actl{#1}{#2}}}}
\newcommand{\TLbinl}[2]{\mathbin{\TLopl{#1}{#2}}}
\newcommand{\CTLopl}[3]{\operatorname{\TLfont{#1\actl{#2}{#3}}}}
\newcommand{\TLoplr}[3]{\operatorname{\TLfont{\actlr{#1}{#2}{#3}}}}
\newcommand{\TLbinlr}[3]{\mathbin{\TLoplr{#1}{#2}{#3}}}
\newcommand{\CTLoplr}[4]{\operatorname{\TLfont{#1\actlr{#2}{#3}{#4}}}}

% path quantifiers (future/past)
\newcommand{\A}{\TLop{A}}
\newcommand{\E}{\TLop{E}}
\newcommand{\pA}{\TLop{\hat A}}
\newcommand{\pE}{\TLop{\hat E}}

% future, globally, until, weak until
\newcommand{\F}{\TLop{F}}
\newcommand{\Fl}[1]{\TLopl{F}{#1}}
\newcommand{\Flr}[2]{\TLoplr{F}{#1}{#2}}
\newcommand{\AF}{\CTLop{A}{F}}
\newcommand{\AFl}[1]{\CTLopl{A}{F}{#1}}
\newcommand{\AFlr}[2]{\CTLoplr{A}{F}{#1}{#2}}
\newcommand{\EF}{\CTLop{E}{F}}
\newcommand{\EFl}[1]{\CTLopl{E}{F}{#1}}
\newcommand{\EFlr}[2]{\CTLoplr{E}{F}{#1}{#2}}
\newcommand{\pAF}{\CTLop{\hat A}{F}}
\newcommand{\pAFl}[1]{\CTLopl{\hat A}{F}{#1}}
\newcommand{\pAFlr}[2]{\CTLoplr{\hat A}{F}{#1}{#2}}
\newcommand{\pEF}{\CTLop{\hat E}{F}}
\newcommand{\pEFl}[1]{\CTLopl{\hat E}{F}{#1}}
\newcommand{\pEFlr}[2]{\CTLoplr{\hat E}{F}{#1}{#2}}
\newcommand{\G}{\TLop{G}}
\newcommand{\Gl}[1]{\TLopl{G}{#1}}
\newcommand{\Glr}[2]{\TLoplr{G}{#1}{#2}}
\newcommand{\AG}{\CTLop{A}{G}}
\newcommand{\AGl}[1]{\CTLopl{A}{G}{#1}}
\newcommand{\AGlr}[2]{\CTLoplr{A}{G}{#1}{#2}}
\newcommand{\EG}{\CTLop{E}{G}}
\newcommand{\EGl}[1]{\CTLopl{E}{G}{#1}}
\newcommand{\EGlr}[2]{\CTLoplr{E}{G}{#1}{#2}}
\newcommand{\pAG}{\CTLop{\hat A}{G}}
\newcommand{\pAGl}[1]{\CTLopl{\hat A}{G}{#1}}
\newcommand{\pAGlr}[2]{\CTLoplr{\hat A}{G}{#1}{#2}}
\newcommand{\pEG}{\CTLop{\hat E}{G}}
\newcommand{\pEGl}[1]{\CTLopl{\hat E}{G}{#1}}
\newcommand{\pEGlr}[2]{\CTLoplr{\hat E}{G}{#1}{#2}}
\newcommand{\U}{\TLop{U}}
\newcommand{\Ul}[1]{\TLopl{U}{#1}}
\newcommand{\Ulr}[2]{\TLoplr{U}{#1}{#2}}
\newcommand{\AU}{\CTLop{A}{U}}
\newcommand{\AUl}[1]{\CTLopl{A}{U}{#1}}
\newcommand{\AUlr}[2]{\CTLoplr{A}{U}{#1}{#2}}
\newcommand{\EU}{\CTLop{E}{U}}
\newcommand{\EUl}[1]{\CTLopl{E}{U}{#1}}
\newcommand{\EUlr}[2]{\CTLoplr{E}{U}{#1}{#2}}
\newcommand{\pAU}{\CTLop{\hat A}{U}}
\newcommand{\pAUl}[1]{\CTLopl{\hat A}{U}{#1}}
\newcommand{\pAUlr}[2]{\CTLoplr{\hat A}{U}{#1}{#2}}
\newcommand{\pEU}{\CTLop{\hat E}{U}}
\newcommand{\pEUl}[1]{\CTLopl{\hat E}{U}{#1}}
\newcommand{\pEUlr}[2]{\CTLoplr{\hat E}{U}{#1}{#2}}
\newcommand{\W}{\TLop{W}}
\newcommand{\Wl}[1]{\TLopl{W}{#1}}
\newcommand{\Wlr}[2]{\TLoplr{W}{#1}{#2}}
\newcommand{\AW}{\CTLop{A}{W}}
\newcommand{\AWl}[1]{\CTLopl{A}{W}{#1}}
\newcommand{\AWlr}[2]{\CTLoplr{A}{W}{#1}{#2}}
\newcommand{\EW}{\CTLop{E}{W}}
\newcommand{\EWl}[1]{\CTLopl{E}{W}{#1}}
\newcommand{\EWlr}[2]{\CTLoplr{E}{W}{#1}{#2}}
\newcommand{\pAW}{\CTLop{\hat A}{W}}
\newcommand{\pAWl}[1]{\CTLopl{\hat A}{W}{#1}}
\newcommand{\pAWlr}[2]{\CTLoplr{\hat A}{W}{#1}{#2}}
\newcommand{\pEW}{\CTLop{\hat E}{W}}
\newcommand{\pEWl}[1]{\CTLopl{\hat E}{W}{#1}}
\newcommand{\pEWlr}[2]{\CTLoplr{\hat E}{W}{#1}{#2}}
\newcommand{\wF}{\TLop{\widetilde F}}
\newcommand{\wFl}[1]{\TLopl{\widetilde F}{#1}}
\newcommand{\wFlr}[2]{\TLoplr{\widetilde F}{#1}{#2}}
\newcommand{\AwF}{\CTLop{A}{\widetilde F}}
\newcommand{\AwFl}[1]{\CTLopl{A}{\widetilde F}{#1}}
\newcommand{\AwFlr}[2]{\CTLoplr{A}{\widetilde F}{#1}{#2}}

% next
\newcommand{\X}{\TLop{X}}
\newcommand{\AX}{\CTLop{A}{X}}
\newcommand{\EX}{\CTLop{E}{X}}
\newcommand{\pAX}{\CTLop{\hat A}{X}}
\newcommand{\pEX}{\CTLop{\hat E}{X}}
\newcommand{\wX}{\TLop{\widetilde X}}
\newcommand{\AwX}{\CTLop{A}{\widetilde X}}


% hybrid operators
\newcommand{\fix}{\operatorname{\downarrow}}
\newcommand{\jump}{\operatorname{@\!}}

% logic name
%\newcommand{\huctl}{HUCTL$_\text{P}$\xspace}:

%% Definitions

%% General

\newcommand{\var}{\mathit{Var}}
\newcommand{\true}{\mathit{true}}
\newcommand{\false}{\mathit{false}}

%% Notion of time

\newcommand{\future}{{}^\vartriangleright}
\newcommand{\past}{{}^\vartriangleleft}
\newcommand{\withTime}[1]{{}^{#1}}

%% Transition System

\newcommand{\tsS}{\mathit{S}}
\newcommand{\tsT}{\mathit{T}}
\newcommand{\tsAP}{\mathit{AP}}
\newcommand{\tsL}{\mathit{L}}

\DeclareAcronym{TS}{
	short = TS,
	long = transition system
}

\newcommand{\tsTransition}{\rightarrow}

%% Direction Transition System

\newcommand{\dtsS}{\tsS}
\newcommand{\dtsT}{\tsT}
\newcommand{\dtsDir}{\mathit{Dir}}
\newcommand{\dtsAP}{\tsAP}
\newcommand{\dtsL}{\tsL}
\newcommand{\tD}[2]{\ensuremath{\mathcal{D}(#1, #2)}}

\newcommand{\dtsTuple}{\ensuremath{(\dtsS, \dtsDir, \dtsT, \dtsAP, \dtsL)}}

\newcommand{\dtsTransition}[1]{\stackrel{#1}{\rightarrow}}

\DeclareAcronym{DTS}{
	short = DTS, 
	long = direction transition system
}

\newcommand{\dtsSlocal}{\dtsS^\bullet}
\newcommand{\dtsSborder}{\dtsS^{\mathrel{\text{\ooalign{\hss$\bullet$\hss\cr$\bigcirc$}}}}}
\newcommand{\dtsSremote}{\dtsS^\bigcirc}

%% HUCTLp

\newcommand{\ctlOp}[1]{\operatorname{\mathbf{#1}}}

% Path quantifiers

\newcommand{\ctlE}{\ctlOp{E}}
\newcommand{\ctlA}{\ctlOp{A}}

% State quantifiers

\newcommand{\ctlDir}[1]{{}_{#1}}

\newcommand{\ctlX}{\ctlOp{X}}
\newcommand{\dctlX}[1]{\ctlOp{X}\ctlDir{#1}}
\newcommand{\dctlXW}[1]{\ctlOp{\widetilde{X}}\ctlDir{#1}}
\newcommand{\ctlF}{\ctlOp{F}}
\newcommand{\dctlF}[1]{\ctlDir{#1}\ctlOp{F}}
\newcommand{\dctlFW}[1]{\ctlDir{#1}\ctlOp{\widetilde{F}}}
\newcommand{\ctlG}{\ctlOp{G}}
\newcommand{\dctlG}[1]{\ctlDir{#1}\ctlOp{G}}
\newcommand{\ctlU}{\ctlOp{U}}
\newcommand{\dctlUl}[1]{\ctlDir{#1}\ctlOp{U}}
\newcommand{\dctlUlr}[2]{\ctlDir{#1}\ctlOp{U}_{#2}}
\newcommand{\ctlW}{\ctlOp{W}}
\newcommand{\dctlWl}[1]{\ctlDir{#1}\ctlOp{W}}
\newcommand{\dctlWlr}[2]{\ctlDir{#1}\ctlOp{W}_{#2}}

\newcommand{\hctlBind}[1]{\fix #1 : }
\newcommand{\hctlAt}[1]{\jump #1 : }
\newcommand{\hctlExists}[1]{\exists #1 : }
\newcommand{\hctlForall}[1]{\forall #1 : }

\DeclareAcronym{HUCTLp}{
	short = HUCTL$_\text{P}$,
	long = hybrid computation tree logic with past
}

\DeclareAcronym{CTL}{
	short = CTL , 
	long = computation tree logic
}

%% Parametrised Direction Transition System

\DeclareAcronym{PDTS}{
	short = PDTS, 
	long = parametrised direction transition system
}

\newcommand{\tP}[2]{\ensuremath{\mathcal{P}(#1, #2)}}
\newcommand{\pdtsT}{\hat{T}}
\newcommand{\pdtsP}{\mathcal{P}}
\newcommand{\pdtsTuple}{\ensuremath{(\pdtsP, \dtsS, \dtsDir, \pdtsT, \dtsAP, \dtsL)}}
\newcommand{\pdtsTransition}[2]{\stackrel{#1}{\rightarrow}_#2}

%% Parameter sets

\newcommand{\ttrue}{\texttt{tt}}
\newcommand{\ffalse}{\texttt{ff}}

%% Assumptions

\newcommand{\assume}{\ensuremath{\mathcal{A}}}
\newcommand{\assumeT}{\ensuremath{\mathcal{A}^\top}}
\newcommand{\assumeF}{\ensuremath{\mathcal{A}^\bot}}