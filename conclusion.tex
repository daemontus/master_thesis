In this work, we presented an efficient distributed fixed point algorithm for solving the parameter synthesis problem for \acf{PDTS} with properties specified using the \acf{HUCTLp}. The algorithm works in a semi-symbolic manner, with explicit state space and symbolic parameter space representations, relying on an appropriate solver for deciding and simplifying the parameter sets. 

\ac{HUCTLp} is a more expressive extension of \ac{CTL}, which allows specification of various interesting properties, such as strongly connected components, cycles or directed runs. We provide a detailed discussion of its semantics and its relationship with \ac{CTL}.

We also provide an implementation of the above mentioned algorithm, which is optimised for multi-core usage. The implementation is freely available as part of the Pithya parameter synthesis tool. As a modelling framework, the implementation provides a module for working with ordinary differential equation based models. However, the core algorithm is completely model agnostic. We also provide a bridge to the Microsoft Z3 solver and various domain-specific optimised solvers.

For this implementation, we provide a case study which explores the terminal components and terminal cycles of two well know models from systems biology. We also provide a scalability analysis which shows that the algorithm is able to utilise provided computational resources.

As future work, we would like to extend the implementation with distributed computation capabilities, since the main framework is already prepared for this, only an appropriate \texttt{Communicator} is needed. Other possible research direction would be to design a more general, fixed point computation framework, which can then be used to implement other common algorithms, such as more efficient component detection. One can also consider a cloud oriented approach to the current fixed point algorithm, relying on stream processing. Finally, the implementation would greatly benefit from more domain specific solvers and on-the-fly compilation of models, which would speed up the parameter set related operations and state space generation.