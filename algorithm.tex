In this chapter, we describe the distributed algorithm used for the parameter synthesis computation on \ac{PDTS} and \ac{HUCTLp} properties.

\section{Assumption semantics}

In order to represent the intermediate results during the computation and accommodate for the distributed nature of the \ac{PDTS} fragments, we introduce the notion of assumptions and related assumption semantic function for the \ac{HUCTLp} formulae.

\subsection{Assumption function}

We define an assumption function for a \ac{PDTS} fragment $i$ as $\assume_i(\psi, h, s) = (\Phi^\top, \Phi^\bot)$, meaning that the \ac{HUCTLp} property $\psi$ is assumed to hold in state $s \in \dtsS$ and valuation $h : \var \to \dtsS$ under parameter valuation $p \in \pdtsP$ such that $p \models \Phi^\top$ and symmetrically, $\psi$ is assumed \emph{not} to hold for parameter valuations $p \models \Phi^\bot$. Collectively, assumption functions of all fragments represent the total knowledge about the system accumulated so far.

We will write $\assumeT_i(\psi, h, s)$ and $\assumeF_i(\psi, h, s)$ to denote an assumption function which returns just the first parameter constrain $\Phi^\top$ or second parameter constrain $\Phi^\bot$ respectively. Finally, when the fragment identifier is clear from context, we can omit the subscript $i$.

We say that an assumption $\assume(\varphi, s, h) = (\Phi^\top, \Phi^\bot)$ about the \ac{PDTS} $\mathcal{K}$ is \emph{valid} when for all $p$ holds that:

\begin{align*}
	(p \models \Phi^\top \Rightarrow (\mathcal{K}_p, s, h) \models \varphi) \land (p \models \Phi^\bot \Rightarrow (\mathcal{K}_p, s, h) \not\models \varphi) 
\end{align*}

From now on, we will only consider valid assumption functions unless stated otherwise. Furthermore, from this definition, we can conclude several important properties of valid assumption functions:

\begin{itemize}
	\item Assumption implies satisfaction, but satisfaction does not necessarily imply assumption.
	\item Parameter constrain $\assumeT(\psi, s, h) \land \assumeF(\psi, s, h)$ can never be satisfiable - i.e. $\psi$ cannot be assumed to be valid and invalid at the same time.
	\item Parameter constrain $\neg\assumeT(\psi, s, h) \land \neg\assumeF(\psi, s, h)$ can be satisfiable. This situation signifies that for some parameter valuations the result is yet \emph{unknown}. We will denote this parameter constrains as $\assumeU(\psi, s, h)$.
	\item These intuitive equalities hold only when the set of unknown parameter valuations is empty:
	
	\begin{align*}
		\assumeT(\psi, s, h) & \not\equiv \neg \assumeF(\psi, s, h) \\
		\assumeT(\psi, s, h) & \not\equiv \neg \assumeT(\neg \psi, s, h)
	\end{align*}
	
	The expression $\neg \assumeF(\psi, s, h)$ represents not only parameters for which $\psi$ is satisfied, but also parameters for which the result is unknown. Similarly, in the second case, $\neg \assumeT(\neg \psi, s, h)$ denotes not only parameters for which the $\neg \psi$ is not satisfied (hence $\psi$ is satisfied) but also unknown parameter valuations.
	
\end{itemize}

We use $\assume_0$ to denote an assumption function which has all results initialised to $(\ffalse, \ffalse)$, hence the set of the valid and invalid parameter valuations is empty and the set of the unknown parameter valuations is the whole parameter universe.

Finally, we assume a partial ordering on the set of all possible assumption functions over \ac{PDTS} fragment $\mathcal{K}_i$ based on the standard set inclusion relation. Formally:

\begin{align*}
	\assume_1 \leq \assume_2 \iff & \forall \psi, s, h : \assumeU_2(\psi, s, h) \subseteq \assumeU_1(\psi, s, h) \\
\end{align*}

Under such ordering, the set of all assumption functions forms a complete lattice. This ordering captures the intuitive understanding of computation progress, because an assumption function $\assume_2$ is bigger than $\assume_1$ only if the complete knowledge about the system is higher in $\assume_2$. Also observe that the highest element of every chain is an assumption function with no unknown parameters while the lowest element is the $\assume_0$. 

\subsection{Semantic function}
\label{sec:semantic}

In order to compute the maximal valid assumption function for a \ac{PDTS} fragment $\mathcal{K}_i = \pdtsTuple$, we define a semantic function $\mathcal{C}_\mathcal{K}(\assume_{in}) = \assume_{out}$ which (if possible) increases the given assumption (with respect to the ordering) while preserving assumption validity.

In this definition, we do not use the full operator set of \ac{HUCTLp}, but rather the simplified set of operators defined in section \ref{sec:huctlEnd}. The definition uses a special assignment operator $\assume(\psi, s, h) \Leftarrow \Phi$ which means that the parameter constrain $\Phi$ is added to the current set of assumptions, formally:

\begin{align*}
	\assume(\psi, s, h) \Leftarrow \Phi \equiv \assume(\psi, s, h) \gets \assume(\psi, s, h) \lor \Phi
\end{align*}

Also observe that this operation is monotonic with respect to the assumption function ordering, that is, the set of unknown parameter valuations can only decrease by applying $\Leftarrow$.

The definition if the semantic function itself is divided into several parts, based on the \ac{HUCTLp} operators it deals with. Each part declaratively defines the new assumption function $\assume_{out}$ based on the given $\assume_{in}$.

\begin{figure}
	\begin{align*}
	\assume_{out}(true, s, h) & \Leftarrow (\ttrue, \ffalse) \\
	\assume_{out}(p, s, h) & \Leftarrow 
	\begin{cases}
	(\ttrue, \ffalse) & p \in \dtsL(s) \\
	(\ffalse, \ttrue) & \texttt{otherwise} \\
	\end{cases}
	\\
	\assume_{out}(x, s, h[ x \mapsto s' ]) & \Leftarrow
	\begin{cases}
	(\ttrue, \ffalse) & s = s' \\
	(\ffalse, \ttrue) & \texttt{otherwise}
	\end{cases}
	\\
	\end{align*}
	
	\caption{Definition of semantic function $\mathcal{C}$ for atomic propositions, $p \in \dtsAP$ is an atomic proposition and $x \in Var$ is a state variable.}
	\label{fig:assProp}
\end{figure}

First part, presented in Figure \ref{fig:assProp}, describes the initial assumptions of the system. Please observe that in all cases, no unknown parameters are present. This naturally follows from the fact that the validity of the atomic formulae depends solely on the properties of the system and the variable valuation, and hence can be resolved unambiguously.

\begin{figure}
	\hspace*{-1.0cm}
\scalebox{0.92}{\parbox{.5\linewidth}{%
		\begin{align*}
		\assumeT_{out}(\withTime{t}\ctlE\dctlX{\chi} \varphi, s, h)  \Leftarrow & \bigvee_{s' \in \dtsSrelevant} \withTime{t}trans(s,s') \land [\withTime{t}dir(s, \chi, s') \land \assumeT_{in}(\varphi, s', h)] \\
		\assumeF_{out}(\withTime{t}\ctlE\dctlX{\chi} \varphi, s, h) \Leftarrow & \bigwedge_{s' \in \dtsSrelevant} \neg\withTime{t}trans(s,s') \lor [\withTime{t}dir(s, \chi, s') \Rightarrow \assumeF_{in}(\varphi, s', h)] \\
		%
		\assumeT_{out}(\withTime{t}\ctlA\dctlX{\chi} \varphi, s, h) \Leftarrow & \bigwedge_{s' \in \dtsSrelevant} \neg \withTime{t}trans(s,s') \lor [\withTime{t}dir(s, \chi, s') \land \assumeT_{in}(\varphi, s', h)] \\
		\assumeF_{out}(\withTime{t}\ctlA\dctlX{\chi} \varphi, s, h)  \Leftarrow & \bigvee_{s' \in \dtsSrelevant} \withTime{t}trans(s,s') \land [ \withTime{t}dir(s, \chi, s') \Rightarrow \assumeF_{in}(\varphi, s', h)] \\
		%
		\assumeT_{out}(\withTime{t}\ctlE\dctlFW{\chi} \varphi, s, h)  \Leftarrow &\ \assumeT_{in}(\varphi, s, h) \lor \\
		& \bigvee_{s' \in \dtsSrelevant} \withTime{t}trans(s,s') \land [\withTime{t}dir(s, \chi, s') \Rightarrow \assumeT_{in}(\withTime{t}\ctlE\dctlFW{\chi} \varphi, s', h) ] \\
		\assumeF_{out}(\withTime{t}\ctlE\dctlFW{\chi} \varphi, s, h)  \Leftarrow &\ \assumeF_{in}(\varphi, s, h) \land \\
		& \bigwedge_{s' \in \dtsSrelevant} \neg\withTime{t}trans(s,s') \lor [\withTime{t}dir(s, \chi, s') \land \assumeF_{in}(\withTime{t}\ctlE\dctlFW{\chi} \varphi, s', h) ] \\
		%
		\assumeT_{out}(\withTime{t}\ctlA\dctlFW{\chi} \varphi, s, h)  \Leftarrow &\ \assumeT_{in}(\varphi, s, h) \lor \\
		& \bigwedge_{s' \in \dtsSrelevant} \neg\withTime{t}trans(s,s') \lor [\withTime{t}dir(s, \chi, s') \Rightarrow \assumeT_{in}(\withTime{t}\ctlA\dctlFW{\chi} \varphi, s', h) ] \\
		\assumeF_{out}(\withTime{t}\ctlA\dctlFW{\chi} \varphi, s, h)  \Leftarrow &\ \assumeF_{in}(\varphi, s, h) \land \\
		& \bigvee_{s' \in \dtsSrelevant} \withTime{t}trans(s,s') \land [\withTime{t}dir(s, \chi, s') \land \assumeF_{in}(\withTime{t}\ctlA\dctlFW{\chi} \varphi, s', h) ] \\
		%
		\assumeT_{out}(\withTime{t}\ctlE[ \varphi_1 \dctlUl{\chi} \varphi_2], s, h) \Leftarrow &\ \assumeT_{in}(\varphi_2, s, h) \lor \big[ \assumeT_{in}(\varphi_1, s, h) \land \\
		& \bigvee_{s' \in \dtsSrelevant} \withTime{t}trans(s,s') \land [\withTime{t}dir(s, \chi, s') \land \assumeT_{in}(\withTime{t}\ctlE[ \varphi_1 \dctlUl{\chi} \varphi_2], s', h)] \big] \\
		\assumeF_{out}(\withTime{t}\ctlE[ \varphi_1 \dctlUl{\chi} \varphi_2], s, h) \Leftarrow &\ \assumeF_{in}(\varphi_2, s, h) \land \big[ \assumeF_{in}(\varphi_1, s, h) \lor \\
		& \bigwedge_{s' \in \dtsSrelevant} \neg\withTime{t}trans(s,s') \lor [\withTime{t}dir(s, \chi, s') \Rightarrow \assumeF_{in}(\withTime{t}\ctlE[ \varphi_1 \dctlUl{\chi} \varphi_2], s', h)] \big] \\
		%
		\assumeT_{out}(\withTime{t}\ctlA[ \varphi_1 \dctlUl{\chi} \varphi_2], s, h) \Leftarrow &\ \assumeT_{in}(\varphi_2, s, h) \lor \big[ \assumeT_{in}(\varphi_1, s, h) \land \\
		& \bigwedge_{s' \in \dtsSrelevant} \neg\withTime{t}trans(s,s') \lor [\withTime{t}dir(s, \chi, s') \land \assumeT_{in}(\withTime{t}\ctlA[ \varphi_1 \dctlUl{\chi} \varphi_2], s', h)] \big] \\
		\assumeF_{out}(\withTime{t}\ctlA[ \varphi_1 \dctlUl{\chi} \varphi_2], s, h) \Leftarrow &\ \assumeF_{in}(\varphi_2, s, h) \land \big[ \assumeF_{in}(\varphi_1, s, h) \lor \\
		& \bigvee_{s' \in \dtsSrelevant} \withTime{t}trans(s,s') \land [\withTime{t}dir(s, \chi, s') \Rightarrow \assumeF_{in}(\withTime{t}\ctlA[ \varphi_1 \dctlUl{\chi} \varphi_2], s', h)] \big] \\
		\end{align*}
}}
	\caption{Definition of semantic function $\mathcal{C}$ for temporal operators.}
	\label{fig:assTemporal}
\end{figure}

In the second part, shown in Figure \ref{fig:assTemporal}, we provide the semantics for the temporal operators. As we can see, the differences between operators and their positive and negative semantics are often very subtle, but crucial.

First of all, let us observe that due to the negation inequality, we can't simply handle the negative assumptions as negations of the positive ones. Second, we see that the observations about the nature of weak operators made in the section \ref{sec:weak} are still valid—that is, one can clearly observe the strict ($\land$) and weak($\Rightarrow$) direction requirement in all definitions.

\begin{figure}
	\begin{align*}
	\assumeT_{out}(\varphi_1 \land \varphi_2, s, h) \Leftarrow &\ \assumeT_{in}(\varphi_1, s, h) \land \assumeT_{in}(\varphi_2, s, h) \\
	\assumeF_{out}(\varphi_1 \land \varphi_2, s, h) \Leftarrow &\ \assumeF_{in}(\varphi_1, s, h) \lor \assumeF_{in}(\varphi_2, s, h) \\
	\assumeT_{out}(\hctlExists{x} \varphi, s, h) \Leftarrow &\ \bigvee_{s' \in S} \assumeT_{in}(\varphi, s, h[x \mapsto s']) \\
	\assumeF_{out}(\hctlExists{x} \varphi, s, h) \Leftarrow &\ \bigwedge_{s' \in S} \assumeF_{in}(\varphi, s, h[x \mapsto s']) \\
	\assume_{out}(\neg \varphi, s, h) \Leftarrow &\ (\assumeF_{in}(\varphi, s, h), \assumeT_{in}(\varphi, s, h)) \\	
	\assume_{out}(\hctlBind{x} \varphi, s, h) \Leftarrow &\ \assume_{in}(\varphi, s, h[x \mapsto s]) \\
	\assume_{out}(\hctlAt{x} \varphi, s, h[x \mapsto s']) \Leftarrow &\ \assume_{in}(\varphi, s', h[x \mapsto s']) \\
	\end{align*}
	
	\caption{Definition of semantic function $\mathcal{C}$ for hybrid and boolean operators.}
	\label{fig:assHybrid}
\end{figure}

Finally, the semantics of the remaining boolean and hybrid operators are described in Figure \ref{fig:assHybrid}.

Notice that except for the \emph{at} operator, the semantic function uses only the \emph{relevant} states of the \ac{PDTS} fragment. Indeed, the \emph{at} operator is fundamentally designed to be able to access any state in the system and hence can't be subject to such requirement.

\subsection{Semantic function fixed point}

As we can see, the semantic function computed according to the rules presented in this section is clearly monotonic (thanks to the $\Leftarrow$ operator) with respect to the complete lattice formed by the assumption functions of $\mathcal{K}$. Therefore according to the Knaster-Tarski theorem \cite{tarski1955}, $\mathcal{C}$ has a least fixed point which can be computed by repeated application of $\mathcal{C}$.

\subsection{Semantic function validity}

Even though the monotonicity of the semantic function is rather obvious, the validity of the resulting assumption function does not have to be entirely clear. Especially for the temporal operators, which are now defined recursively based on the \ac{PDTS} transitions rather than using the infinite runs.

\begin{lemma}
	\label{lemma:semantic}
	Given a valid assignment $\assume_{in}$, the semantic function $\mathcal{C}(\assume_{in})$ produces a valid assignment.
\end{lemma}

\textbf{Proof} First, we consider the validity of the atomic proposition assumptions and the hybrid operator assumptions as obvious, since they can be almost effortlessly translated to their direct definitions.

Next, we show a full validity proof for the positive assumption of the $\withTime{t}\ctlA\dctlUl{\chi}$ operator—one of the most complex temporal operators. Proofs for other operators (and their negative counterparts) are very similar and therefore we won't list them in detail.

\begin{itemize}
	\item Assume all current $\assumeT_{in}$ and $\assumeF_{out}$ are valid and we are considering $ \psi = \withTime{t} \ctlA[ \varphi_1 \dctlUl{\chi} \varphi_2 ]$.
	\item For a contradiction, let us assume that after application of $\mathcal{C}$, there exists a parameter valuation $p \in \pdtsP$ such that $p \models \assumeT_{out}(\psi, s, h)$ and $(\mathcal{K}_p, s, h) \not\models \psi$ for some $s$ and $h$.
	\item According to the definition of $\mathcal{C}$, this can happen in two cases:
	\begin{itemize}
		\item $p \models \assumeT_{in}(\varphi_2, s, h)$ - this is in direct contradiction with the operator definition, since for all runs $\pi$ there exists $i = 0$ such that $(\mathcal{K}_p, \pi_\dtsS(i), h) \models \varphi_2$ because $\pi_\dtsS(0) = s$. Therefore $(\mathcal{K}_p, s, h) \models \psi$ (the direction requirement is trivially satisfied since there is no $j$ smaller than $0$).
		\item $p \models \big[ \assumeT_{in}(\varphi_1, s, h) \land \bigwedge_{s' \in \dtsS} \neg\withTime{t}trans(s,s') \lor [\withTime{t}dir(s, \chi, s') \land \assumeT_{in}(\withTime{t}\ctlA[ \varphi_1 \dctlUl{\chi} \varphi_2], s', h)] \big]$. Hence not only $(\mathcal{K}_p, s, h) \models \varphi_1$, but also for all $s'$ either the transition from $s$ is not present, or if it is present, it satisfies the direction requirement, and $p \models \assumeT_{in}(\withTime{t}\ctlA[ \varphi_1 \dctlUl{\chi} \varphi_2], s', h)$. Thanks to this fact, we see that every infinite run starting in $s$ has to use one of these transitions. Therefore the only case when $(\mathcal{K}_p, s, h) \not\models \psi$ is when either the assumption $\assumeT_{in}(\varphi_1, s, h)$ or some of the assumptions $\assumeT_{in}(\withTime{t}\ctlA[ \varphi_1 \dctlUl{\chi} \varphi_2], s', h)$ are not valid, which is a contradiction with our original assumption.
	\end{itemize}
\end{itemize}
\qed

\section{Main algorithm}

In this section, we present the main algorithm which relies on the previously defined assumption function and the assumption fixed point.

\subsection{Environment and data structures}

Before we describe the main algorithm, we have to introduce our assumptions about the environment the algorithm will be executed in.

The algorithm presented in this work assumes a distributed environment of $N$ independent, reliable agents connected using reliable communication channels. In practice, such model is suitable to model multi core machines and small to medium computational clusters where fault tolerance is not strictly necessary.

We assume that each agent has local access to the following data structures:

\begin{itemize}
	\item Total number of communicating agents $N$;
	\item A unique agent identifier $id \in \{ 1, \ldots, N \}$;
	\item A partition function $f : \dtsS \to \{1, \ldots, N \}$;
	\item A \ac{PDTS} fragment $\mathcal{K}_i$;
	\item A \ac{HUCTLp} formula $\varphi$;
%	\item An initial parameter constrain $\Phi_I$;
\end{itemize}

Note that not all of these structures have to be represented explicitly. For example, the \ac{PDTS} fragment and the partition function is usually computed on-the-fly.

\subsection{Algorithm pseudocode}

The Algorithm \ref{alg:main} starts with a fully unknown assignment and iteratively computes new fixed point assumptions. As soon as the fixed point is reached, a round of communication is performed to ensure all remaining fragments are notified about the updated assumptions. At this point, information from other fragments is also received. As soon as no new information can be gained using this procedure, the algorithm proceeds to mark unknown parameter valuations as unsatisfied, assuming all sub-formulae have been successfully computed.

\begin{algorithm}
	\begin{algorithmic}[1]
		\Function{ParameterSynthesis}{fragment $\mathcal{K}_{id}$, property $\varphi$}
		\State $\assume_{id} \gets \assume_0$
		\Repeat
		\Repeat
		\Repeat
		\State $\mathcal{T} \gets \assume_{id}$
		\State $\assume_{id} \gets \mathcal{C}(\mathcal{T})$
		\Until{$\mathcal{T} = \assume_{id}$}
		\State $\Call{Communicate}{\assume_{id}}$
		\Until{\Call{TerminationDetection}{}}
		\State $\Call{IncreaseCycles}{\assume_{id}, \varphi}$
		\Until{\Call{TerminationDetection}{}}
		\State $\mathcal{F}_i(s) \gets \assumeT_{id}(\varphi, s, h_0)$
		\EndFunction
		\Function{IncreaseCycles}{assumption function $\assume$, property $\varphi$}
			\State for $s \in \dtsSlocal : Done(h, s) \gets \ttrue $
			\ForAll{$\psi \in sub(\varphi)$}
				\State $Done(h, s) = Done(h, s) \land \neg \Call{IncreaseCycles}{\mathcal{A}, \psi}(h, s)$
			\EndFor
			%\State for $s \in \dtsSlocal : Unknown(h, s) \gets \neg \assumeT(\varphi, h, s) \land \neg \assumeF(\varphi, h, s)$		
			\State for $s \in \dtsSlocal : \assumeF(\varphi, h, s) \Leftarrow \assumeU(\varphi, h, s) \land Done(h,s)$
			\State \Return $Unknown$
		\EndFunction
		\Function{Communicate}{assumption function $\assume$}
			\ForAll{$s \in \dtsSlocal_{id}$}
				\ForAll{$i < N$ such that $s \in \dtsSborder_i$}
					\State for all $\psi,h : \assume_i(\psi, h, s) \Leftarrow \assume_{id}(\psi, h, s)$
				\EndFor
				\ForAll{$i < N$ and $\psi : (\hctlAt{x} \psi) \in cl(\varphi) $}
					\State for all $h, s : \assume_i(\psi, h, s) \Leftarrow \assume_{id}(\psi, h, s)$
				\EndFor
			\EndFor
		\EndFunction
	\end{algorithmic}	

	\caption{Main fixed point algorithm.}
	\label{alg:main}
\end{algorithm}

Here, the $\textsc{TerminationDetection}$ procedure will return true only if the assumptions have not changed since last termination attempt and no communication is taking place. We don't discuss a specific implementation for this procedure, since termination detection in distributed systems is a well studied problem \cite{termination} and an optimal implementation usually depends on the available communication primitives in our environment.

To simplify the description of the $\textsc{Communication}$ procedure, we assume that each agent can directly update assumptions of other agents. This is of course an unrealistic expectation in a distributed system. In reality, this form of communication assumes a proper message is sent, received and handled by the receiving agent. However, we do not provide an explicit pseudocode for this, since it is, similar to the $\textsc{TerminationDetection}$, a problem which largely depends on the available environment (in a multi-core machine simulating the distributed environment, even a direct update can be possible) and has been efficiently solved before \cite{comm}.

Note that, as we also discussed in the previous section, the \emph{at} operator requires special treatment, due to its ability to access arbitrary states.

Finally, let us observe that the complete solution to the parameter synthesis problem can be constructed from the partial solutions computed by each fragment as follows:

\begin{align*}
	\mathcal{F}(s) = \bigvee_{i \in \{1, \ldots, N \}} \mathcal{F}_i(s)
\end{align*}

\subsection{Correctness}

\subsubsection{\textbf{Termination}}

\begin{theorem} 
	Given a \ac{PDTS} fragment $\mathcal{K}_{id}$ and a \ac{HUCTLp} property $\varphi$, the~Algorithm \ref{alg:main} eventually terminates.
\end{theorem}

\textbf{Proof} First, let us observe that since the least fixed point of $\mathcal{C}(\assume)$ always exists, the condition at line $8$ will also always terminate as soon as this fixed point is reached. Furthermore, since the communication function is also monotonic with respect to the assumption ordering, the condition at line $10$ will also eventually reach a fixed point as soon as no new assumptions can be computed either by applying the semantic function, or communicating with other processes.

Finally, the function $\textsc{IncreaseCycles}$ is also monotonic, and therefore the same argument holds for the condition on line $12$. Eventually, all agents will reach a global fixed point and at that point, the system will terminate. \qed

\subsubsection{\textbf{Partial Correctness}}

Before we get to the main theorem, let us prove several additional lemmas about the whole algorithm that will greatly simplify the correctness proof:

\begin{lemma}
		\label{lemma:comm}
	Procedure \textsc{Communicate} preserves validity of the assumption function.
\end{lemma}

\textbf{Proof} This is almost obvious, since the \textsc{Communicate} procedure will only increase assumptions in border states of fragment $i$ if current fragment $id$ successfully computed said assumption. Therefore, assuming the information for the local states in this fragment is valid (given that the semantic funciton $\mathcal{C}$ preserves assumption validity), the newly assigned information for the border states is also valid.

The exception being the \emph{at} operator, which is always delivered to all fragments, regardless of border states. However, this also adheres to the global \emph{at} semantics. \qed

\begin{lemma}
		\label{lemma:cycles}
	Assuming no value of $\assume_i(\psi, h, s)$ (globally for all $i$) can be increased either by applying $\mathcal{C}$ or by performing $\textsc{Communicate}$, the procedure \textsc{IncreaseCycles} preserves validity of the assumption function.
\end{lemma}

\textbf{Proof} It should be noted that the assumption of the lemma corresponds to the conditions under which is the \textsc{IncreaseCycles} called in the algorithm (the conditions are ensured by the repeat-until loop on lines 4-10). 

Second, the procedure can increase only negative assumptions about the system. Therefore, for a contradiction, let us assume that after application of $\textsc{IncreaseCycles}$ there is a fragment $\mathcal{K}_i$, a parameter valuation $p$, a property $\psi$, a variable assignment $h$ and a state $s$, such that $p \models \assumeF_i(\psi, h, s)$ but $(\mathcal{K}_p, h, s) \models \psi$. 

This can only happen when the assumptions about $p$ are known for all sub-formulae of $\psi$, since the only time when the unknown assumption can increase is when the parameter valuation is marked as $Done$, which means it is not unknown in any of the sub-formulae.

Because assumptions for all sub-formulae are known, the only operators for which this error can occur are the temporal operators. That is because all other operators depend directly on their sub-formulae, and therefore when the assumptions for the sub-formulae are known, the assumption for the formula should also be known when the fixed point is computed. This would imply that the fixed point was not computed properly, contradicting the original assumption of the lemma. 

%Here we should also note that except for the $\hctlAt{x} \varphi$ operator, all operators depend only on other local and border states.  

However, the assumption of the computation and communication fixed point also implies that the only possible case when assumption for such operator can be unknown is in case of a circular dependency between assumptions. Any other case would imply that either the communication has not been handled properly (and therefore we have unknown parameter valuations that should have been resolved using communication) or the fixed point has not been reached yet (and therefore some parameter valuations can still increase due to assumptions about their successors).

Finally, if a parametrisation is unknown due to a circular dependency between assumptions, it can be safely assumed to be negative. That is because such assumption cannot ever increase to a positive value, therefore also the property cannot hold in such state. \qed

\begin{lemma}
	\label{lemma:unknown}
	When the algorithm exits the main cycle and enters line 13, all unknown assumptions for the local states $\assumeU(\psi, h, s)$ are empty.
\end{lemma}

\textbf{Proof} This is easy to show, because in each iteration, the procedure \textsc{IncreaseCycles} increases at least one unknown parameter valuation to the negative value for some state (assuming unknown parameter valuations exist). 

Therefore, assuming there are some unknown assumptions, there must exists the smallest formula $\psi$ such that some of its assumptions are unknown and yet all sub-formulae are already known. These unknown assumptions are then promoted to negative ones by the \textsc{IncreaseCycles} procedure. Hence after a finite number of repetitions, \textsc{IncreaseCycles} increases all unknown parameter valuations to negative ones. \qed

\begin{theorem}
	Algorithm \ref{alg:main} computes the parameter synthesis problem for a given \ac{PDTS} $\mathcal{K}$ and a given \ac{HUCTLp} property $\varphi$.
\end{theorem}

\textbf{Proof} By Lemma \ref{lemma:unknown}, there are no unknown parameter valuations for the local states of each fragment. Therefore an union across all fragments ensures that there are no unknown parameter valuations across the whole state space. Furthermore, since the assumption functions of each fragment are valid (by Lemma 	\ref{lemma:semantic}, \ref{lemma:comm} and \ref{lemma:cycles}), the resulting assumption function is also valid. \qed

\subsection{Notes on the semantic function and complexity}

One notable part of the algorithm that has been intentionally left out of the main pseudocode is how to compute the semantic function (and its fixed point). Similar to the \textsc{TerminationDetection} and the \textsc{Communication} procedures, there are several possible approaches which depend on the nature of the model, the execution environment and other desired properties of the algorithm.  Instead of providing the full pseudocode for this operation, we provide a discussion of several observations and approaches that can be taken when implementing the semantic function fixed point and we refer the reader to the actual implementation of the algorithm for one specific example.

\begin{itemize}
	\item In the definition of the semantic function, assumptions about the temporal operators usually depend on all local and border states. This requirement can be easily reduced to just the actual successors (for some parameter valuation). In transition systems where a state has only a small number of potential successors, this means that the number of assumption dependencies is often not linear in the size of state space, but rather constant or at least logarithmic.
	\item Complexity can be further reduced by memorizing intermediate results of operations. When assumption in state $s$ depends on its $k$ successors, one can build a tree-like structure that can be then updated in logarithmic time when one of the dependencies changes.
	\item Since determining whether a value of an assumption changed or not requires a semantic equality check, which is often a very expensive operation, it can be beneficial to delay such check as long as possible (without introducing additional checks later). An example of such delaying would be to ensure, that the check is performed only when no dependencies of the assumption under question are still being updated (or waiting to be updated). In other words, one can process the dependencies in a "distance-to-proposition" order, ensuring the values are updated only when necessary.
	\item It is crucial for the effectiveness of the algorithm to maintain compact representations of the parameter formulae produced during computation. To this end, instead of using a simple formula solver, one should look for solvers that can also simplify the investigated formulae. However, such operations are usually even harder than standard formula solving. It might be therefore beneficial to employ heuristics that can delay formula simplification when not needed or that can separate parts of the formula that cannot be further simplified.
\end{itemize}

